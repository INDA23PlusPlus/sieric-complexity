\clearpage
\section*{Induktion}

\subsection*{(a)}

Bevisa följande ekvation med hjälp av induktion

\[\sum_{i=1}^{n}{i^{2}}=\frac{n\paren{n+1}\paren{2n+1}}{6}.\]

\begin{alignat*}{2}
  \intertext{
\textbf{Basfall $n=1$:}
  }
  &\VL &&= 1^{2} = 1 \\
  &\HL &&=\frac{1\paren{1+1}\paren{2\paren{1}+1}}{6} \\
  \iff&\HL &&= \frac{6}{6} = 1 = \VL \\
  \intertext{
\textbf{Induktionsantagande:} Antag att ekvationen är sant då $n=p$.
  }
  \intertext{
\textbf{Induktionssteg $n=p+1$:}
  }
  &\VL &&= \sum_{i=1}^{p+1}{i^{2}} \\
  \iff&\VL &&= \paren{p+1}^{2} + \sum_{i=1}^{p}{i^{2}} \\
  \intertext{
Enligt induktionsantagandet är summan ovan lika med
$\frac{p\paren{p+1}\paren{2p+1}}{6}$.
  }
  \iff&\VL &&= \paren{p+1}^{2} + \frac{p\paren{p+1}\paren{2p+1}}{6} \\
  \iff&\VL &&= \frac{6\paren{p+1}^{2} + p\paren{p+1}\paren{2p+1}}{6} \\
  % \iff&\VL &&= \frac{6p^{2}+12p+6 + 2p^{3}+3p^{2}+p}{6} \\
  \iff&\VL &&= \frac{2p^{3}+9p^{2}+13p+6}{6} \\
  \\
  &\HL &&=\frac{\paren{p+1}\paren{p+1+1}\paren{2\paren{p+1}+1}}{6} \\
  \iff&\HL &&=\frac{\paren{p+1}\paren{p+2}\paren{2p+3}}{6} \\
  \iff&\HL &&=\frac{2p^{3}+9p^{2}+13p+6}{6} = \VL \\
  \intertext{
Enligt induktionsprincipen gäller ekvationen för alla $n\geq1$.
  }
\end{alignat*}

\subsection*{(b)}


Bevisa följande ekvation med hjälp av induktion

\[\sum_{j=1}^{n}\paren{2j-1}=n^{2}.\]

\begin{alignat*}{2}
  \intertext{
\textbf{Basfall $n=1$:}
  }
  &\VL &&= 2\paren{1}-1 = 1 \\
  &\HL &&= 1^{2} = 1 = \VL \\
  \intertext{
\textbf{Induktionsantagande:} Antag att ekvationen är sant då $n=p$.
  }
  \intertext{
\textbf{Induktionssteg $n=p+1$:}
  }
  &\VL &&= \sum_{j=1}^{p+1}\paren{2j-1} \\
  \iff&\VL &&= 2\paren{p+1}-1 + \sum_{j=1}^{p}\paren{2j-1} \\
  \intertext{
Enligt induktionsantagandet är summan ovan lika med $p^{2}$.
  }
  \iff&\VL &&= 2\paren{p+1}-1 + p^{2} \\
  \iff&\VL &&= p^{2}+2p+1 \\
  \iff&\VL &&= \paren{p+1}^{2} \\
  \\
  &\HL &&= \paren{p+1}^{2} = \VL \\
  \intertext{
Enligt induktionsprincipen gäller ekvationen för alla $n\geq1$.
  }
\end{alignat*}
